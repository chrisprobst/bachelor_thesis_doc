%!TEX root = bachelor.tex

\pdfbookmark[0]{Abstract}{abstract}
\begin{center} 
\huge Abstract
\end{center}


This thesis concentrates on implementing and evaluating different distribution algorithms for large data transfers in heterogenous P2P networks, especially for incremental transfers, which yields the possibillity for P2P video streaming. The main focus is on $1:n$ scenarios where only one peer has a given data set completely and $n$ peers try to distribute this data set among themselves as fast as possible, though other scenarios are implemented and evaluated as well.

The main problem is the choice of the best distribution algorithm, which should be able to use most of the available network bandwidth of all participating peers. In a traditional client/server system the server uploads the data to all clients sequentially, which means that only the server upload bandwidth is used but none of the client upload bandwidths.

P2P networks in combination with efficient distribution algorithms can help to solve this problem. Those algorithms are always based to the same technique where all participating peers download specific chunks from the peer with the whole data set and upload those chunks to other peers. This way the upload bandwidths of the peers are not idle. The difficulty is the choice of the specific chunks the peers download and the tuning of parameters like chunk count and chunk size. Good algorithms are also flexible enough to adjust themselves to changing network conditions.

-- work in progress