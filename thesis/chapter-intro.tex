%!TEX root = bachelor.tex

\chapter{Introduction}
\section{Problem and Motivation}
Traditional network applications are mostly based on the client\,/\,server model, where the server only responds to client requests and the clients do not know each other. For some server applications more clients also require more upload bandwidth of the server in order to guarantee quality of service.

In case of real-time video streaming, a server should guarantee, that the ability to play a video fluently does not depend on the number of clients watching the video stream in parallel. In consequence of that, a video streaming server can fundamentally only serve a certain number of clients, which are able to watch the video stream in parallel, so this approach has scalability issues.

To improve this imbalance clients could use their own upload capacity to help the server distributing a given data set. These are called \emph{Peer-to-Peer} networks, where every participant is called a \emph{peer} and is connected to other peers in the network.  If all peers have enough upload bandwidth, it is possible to create a Peer-to-Peer network, which is able to distribute a given data set evenly among all peers within a fixed period of time.
\vfill

\section{Objective}
This thesis concentrates on implementing a Peer-to-Peer software, which is able to distribute data sets among any number of peers and never exceeds $2\:T_0$ seconds, where $T_0$ is the time needed for a single transfer between two peers measured in seconds. The application part of the software should be separated from the framework part, so that further development can be based on the same core.

The application part should contain a benchmark module, which is able to simulate different Peer-to-Peer network scenarios and a streaming module, which can be used to distribute data sets among peers in a real network. The framework part should consist of a network module, which handles low-level network access and provides a high-level view of the Peer-to-Peer network, a database module, which offers a generic interface for storing and loading any kind of data, an algorithm module, which implements different distribution models and a monitoring module, which measures certain parameters of the framework.

To determine the efficiency of the framework, different scenarios should be simulated using the benchmark and the monitoring module. In order to get meaningful results, appropriate scenarios should be chosen. The following chapters try to approach these goals.

\section{Structure}
The thesis starts with a theoretical introduction of data distribution in client\,/\,server and Peer-to-Peer networks in Chapter \ref{theory}. Then the architecture of the software is explained in Chapter \ref{architecture} whereupon the framework and application modules are presented in Chapter \ref{module}. The evaluation of the software follows in Chapter \ref{evaluation}. The last Chapter \ref{conclusion} presents future concepts based on the developed framework.